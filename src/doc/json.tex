\documentclass[11pt]{article}
\usepackage{times}
\usepackage{pldoc}
\sloppy
\makeindex

\begin{document}
% This LaTeX document was generated using the LaTeX backend of PlDoc,
% The SWI-Prolog documentation system



\section{json.pl -- graphicalize}

\label{sec:json}

\begin{tags}
    \tag{author}
Paolo Frasconi
    \tag{To be done}
Safety checks, assumption checks, throwing exceptions, performance tuning
\end{tags}

This module contains predicates for converting a deductive database
into a set of graphs, one for each interpretation, using the mapping
procedures described below.

\begin{description}
    \item[Concept] 
Given a relational data base (a collection of ground facts), the
graphicalizer generates an output graph according to one of two
possible strategies.
    \item[Database] 
The database is assumed to be in the following form:
\end{description}

\begin{itemize}
    \item Attributes are either object identifiers or properties.
    \item For each identifier i, there is at least one table t such that i
is the primary key of t. This table is basically the class or
entity-set of i.
    \item For each table, the primary key only consists of identifiers. So
multiway relationships are always represented by a table keyed by
the object identifiers involved in the relationship, augmented
with properties of the relationship.
\end{itemize}

\begin{description}
    \item[Graphicalization Stragegy 1] 
\end{description}

\begin{itemize}
    \item There is a vertex for each identifier (called an i-vertex), a
vertex for each relationship tuple (an r-vertex) and a vertex for
each occurrence of a property value (a p-vertex). p-vertices can be
either associated with predictors or with responses (the latter are
specified by an asterisk * in the signature, see below).
    \item There is an edge between an i-vertex and an r-vertex if i belongs
to the tuple r. There is an edge between an i-vertex and a p-vertex
if p is a property of i. Finally there is an edge between an
r-vertex and a p-vertex if p is a property of r. There are no other
edges.
\end{itemize}

\begin{description}
    \item[Graphicalization Stragegy 2] 
\end{description}

\begin{itemize}
    \item There is a vertex for each identifier (called an i-vertex), and a
vertex for each relationship tuple (an r-vertex). i-vertices are
labeled by a ground fact whose functor is the name of the entity-set
(class) of i and whose arguments are properties of the
class. r-vertices are labeled by a ground fact whose functor is the
name of the relationship and whose arguments are the properties of
the relationship.
    \item There is an edge between an i-vertex and an r-vertex if i belongs
to the tuple r.
\end{itemize}

\begin{description}
    \item[Usage] 
The module provides two fundamental predicates: \predref{attach}{1}, for
loading a data set (variants: \predref{attach}{2} and \predref{attach}{3} for two and
three data files), and \predref{run}{1} for graphicalizing the data set and
saving the output in an extended-gspan file. The extended-gspan
format is defined by the following grammar:

\begin{code}
<egspan> ::= <list-of-graphs>
<list-of-graphs> ::= <graph> | <graph> "\n" <list-of-graphs>
<graph> ::= <header> "\n" <list-of-vertices> "\n" <list-of-edges>
<header> ::= "t # Example:" <graph-identifier>
<graph-identifier> ::= <alphanumeric-string>
<list-of-vertices> ::= <vertex> | <vertex> "\n" <list-of-vertices>
<list-of-edges> ::= <edge> | <edge> "\n" <list-of-edges>
<vertex> ::= "v" <vertex-id> <vertex-label>
<vertex-id> ::= <integer>
<vertex-label> ::= <pair> | <pair> "," <vertex-label>
<pair> ::= <key> ":" <value>
<key> ::= <alphanumeric-string>
<value> ::= <alphanumeric-string>
<edge> ::= "e" <vertex-id> <vertex-id> <edge-label>
<edge-label> ::= <alphanumeric-string>
\end{code}

The first pair in a vertex list has the form meta:$<$kind$>$ where kind
is one of i, r, p, and q (the last two only for stragegy 1). In
strategy 1 there is only one additional pair in the form
label:$<$value$>$, while in stragegy 2 the second pair has the form
label:$<$functor$>$ where $<$functor$>$ is the name of the entity-set
(i-vertices) or the name of the relationship (r-vertices) while the
remaining pairs encode the tuple of properties in the form
$<$property-name$>$:$<$value$>$.
    \item[Syntactical conventions] 
The extensional declarations should use the special predicate
\qpredref{db}{interpretation}{2} whose first argument is an identifier for the
interpration (any Prolog term) and whose second argument is a fact.
    \item[History] 
git log
\end{description}

\vspace{0.7cm}

\begin{description}
    \predicate[det]{atom_json_term}{3}{+Atom, -JSONTerm, +Options}
\nodescription
    \predicate[det]{atom_json_term}{3}{-Text, +JSONTerm, +Options}
Convert between textual representation and a JSON term. In
\textit{write} mode, the option as(Type) defines the output type, which
is one of \const{atom}, \const{string} or \const{codes}.

    \predicate[det]{json_read}{2}{+Stream, -Term}
\nodescription
    \predicate[det]{json_read}{3}{+Stream, -Term, +Options}
Read next JSON value from \arg{Stream} into a Prolog term. \arg{Options}
are:

\begin{description}
    \termitem{null}{NullTerm}
\arg{Term} used to represent JSON \const{null}. Default @(null)
    \termitem{true}{TrueTerm}
\arg{Term} used to represent JSON \const{true}. Default @(true)
    \termitem{false}{FalsTerm}
\arg{Term} used to represent JSON \const{false}. Default @(false)
    \termitem{value_string_as}{Type}
Prolog type used for strings used as value. Default
is \const{atom}. The alternative is \const{string}, producing a
packed string object. Please note that \const{codes} or
\const{chars} would produce ambiguous output and is therefore
not supported.
\end{description}

    \predicate[det,private]{ws}{2}{+Stream, -Next}
\nodescription
    \predicate[private]{ws}{3}{+C0, +Stream, -Next}
Skip white space on the \arg{Stream}, returning the first non-ws
character. Also skips \verb$//$ ... comments.

    \predicate[det,private]{json_write_string}{2}{+Stream, +Text}
Write a JSON string to \arg{Stream}. \arg{Stream} must be opened in a
Unicode capable encoding, typically UTF-8.

    \predicate[det,private]{json_write_indent}{3}{+Stream, +Indent, +TabDistance}
Newline and indent to \arg{Indent}. A Newline is only written if
line_position(\arg{Stream}, Pos) is not 0. Then it writes \arg{Indent} \Sidiv{}
\arg{TabDistance} tab characters and \arg{Indent} mode \arg{TabDistance} spaces.

    \predicate[det]{json_write}{2}{+Stream, +Term}
\nodescription
    \predicate[det]{json_write}{3}{+Stream, +Term, +Options}
Write a JSON term to \arg{Stream}. The JSON object is of the same
format as produced by \predref{json_read}{2}, though we allow for some more
flexibility with regard to pairs in objects. All of Name=Value,
Name-Value and Name(Value) produce the same output. In addition
to the options recognised by \predref{json_read}{3}, we process the
following options are recognised:

\begin{description}
    \termitem{width}{+Width}
Width in which we try to format the result. Too long lines
switch from \textit{horizontal} to \textit{vertical} layout for better
readability. If performance is critical and human
readability is not an issue use Width = 0, which causes a
single-line output.
    \termitem{step}{+Step}
Indentation increnment for next level. Default is 2.
    \termitem{tab}{+TabDistance}
Distance between tab-stops. If equal to Step, layout
is generated with one tab per level.
\end{description}

    \predicate[semidet,private]{json_print_length}{5}{+Value, +Options, +Max, +Len0, +Len}
True if \arg{Len}-\arg{Len0} is the print-length of \arg{Value} on a single line
and \arg{Len}-\arg{Len0} \Sle{} \arg{Max}.

\begin{tags}
    \tag{To be done}
Escape sequences in strings are not considered.
\end{tags}

    \predicate[semidet]{is_json_term}{1}{@Term}
\nodescription
    \predicate[semidet]{is_json_term}{2}{@Term, +Options}
True if \arg{Term} is a json term. \arg{Options} are the same as for
\predref{json_read}{2}, defining the Prolog representation for the JSON
\const{true}, \const{false} and \const{null} constants.
\end{description}


\printindex
\end{document}
