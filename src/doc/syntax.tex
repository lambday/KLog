% This LaTeX document was generated using the LaTeX backend of PlDoc,
% The SWI-Prolog documentation system



\section{syntax.pl: kLog syntax}

\label{sec:syntax}

\begin{tags}
    \tag{author}
Paolo Frasconi
    \tag{To be done}
Safety checks, assumption checks, performance tuning
\end{tags}

This module provides syntactic extensions to Prolog for declaring
signatures. Most of the functionalities in this module are
implemented via \predref{term_expansion}{2}.

\begin{description}
    \item[History] 
see git log
\end{description}

\vspace{0.7cm}

\begin{description}
    \prefixop{signature}{+S}
Declare \arg{S} to be the type signature of a table, according to the following syntax:

\begin{code}
   <signature> ::= <header> [<sig_clauses>]
   <sig_clauses> ::= <sig_clause> | [<sig_clause> <sig_clauses>]
   <sig_clause> ::= <Prolog_clause>
   <header> ::= <sig_name> "(" <args> ")" "::" <level> "."
   <sig_name> ::= <Prolog_atom>
   <args> ::= <arg> | [<arg> <args>]
   <arg> ::= <column_name> [<role_overrider>] "::" <type>
   <column_name> ::= <Prolog_atom>
   <role_overrider> ::= "@" <role>
   <role> ::= <Prolog_atom>
   <type> ::= "self" | <sig_name>
   <level> ::= "intensional" | "extensional"
\end{code}

Arguments are given by default a role that corresponds to their
position in the argument list. The role can be overridden using the
@ syntax, e.g. to represent undirected relations. By default, the
role of the i-th argument is: * functor+i if the argument is an
identifier * functor+t if the argument is a property and t its type

thus, e.g. if a_id and b_id are identifier names, and c_type is a
property name, the following signatures are equivalent:

\begin{code}
signature foo(g_id::gnus,t_id::tnus,c::property)::extensional.
signature foo(g_id@1,t_id@2,c@3::property)::extensional.
\end{code}

Forcing the role can also easily implement undirected edges. Here is an
example from mutagenesis:

\begin{code}
signature atm( atom_id::self,
               element::property,
               quantaval::property,
               chargeval::property).
signature bond( atom_id@b::atom,
                atom_id@b::atom,
                bondtype::property).
\end{code}

More complex cases can be easily conceived, e.g. one can assign the
same role to every atom in a benzene ring but distinguish different
rings in a functional group like ball3 or phenanthrene. See
experiments/mutagenesis for details.

    \predicate[det,private]{expand_signature}{2}{+S, +Level}
Auxiliary predicate for expanding a signature declaration where \arg{S} is
a fact with typed arguments and \arg{Level} is either 'intensional' or
'extensional'. This predicate performs a number of checks and
asserts various signature traits (see below).
\predref{term_expansion}{2}.

    \predicate[det]{domain_traits}{2}{?TraitsSpec, ?TraitsVal}
Query domain traits. \arg{TraitsSpec} is one of:

\begin{itemize}
    \item signatures: list of all signature names
    \item entities: list of all entity signature names
    \item relationships: list of all relationship signature names
\end{itemize}

    \predicate[det]{domain_traits}{0}{}
List on the standard output all the domain traits and signature
traits for all signatures.

    \predicate[det]{domain_traits}{1}{+Stream}
List on output \arg{Stream} all the domain traits and signature
traits for all signatures.

    \predicate[det]{signature_traits}{3}{?Name, ?TraitsSpec, ?TraitsVal}
Query signature traits. \arg{Name} is the signature name. \arg{TraitsSpec} is one of:

\begin{itemize}
    \item kind: either entity or relationship
    \item arity: the number of arguments
    \item level: either intensional (deduced) or extensional
    \item column_types: list of argument types
    \item column_names: list of argument names
    \item column_roles: list of argument roles
    \item relational_arity: the number of non-property arguments
    \item ext_clause: goal for finding all ground tuples for this signature
The value is returned in \arg{TraitsVal}.
\end{itemize}

    \predicate[semidet]{id_position}{2}{+S, -Position}
If \arg{S} is the name of a valid entity signature, unify \arg{Position} with
the position of the identifier in the argument list. Otherwise fail.

    \predicate[semidet]{extract_properties}{3}{+Functor, +Args, -Properties}
If \arg{Functor} is the name of a valid signature whose arity matches the
length of the list \arg{Args}, then unify \arg{Properties} with the list of
items in \arg{Args} that appear at positions of property type. Otherwise
fail.

    \predicate[semidet]{extract_identifiers}{3}{+Functor, +Args, -IDs}
If \arg{Functor} is the name of a valid signature whose arity matches the
length of the list \arg{Args}, then unify \arg{IDs} with the list of items in
\arg{Args} that appear at positions of identifier type. Otherwise fail.

    \predicate[semidet]{extract_references}{2}{+S, -Refs}
If s is the name of a valid signature, then unify \arg{Refs} with the
subsequence of entity types referenced by \arg{S} but $>$self$<$ is replaced
by \arg{S}. If a type is referred to multiple times, it appears multiple
time in \arg{Refs}.

    \predicate[semidet]{is_kernel_point}{2}{+Functor, -YesNo}
If \arg{Functor} is the name of a valid signature then unify \arg{YesNo} with
'yes' iff the signature has declared a neighborhood method. All
graph vertices expanded from this signature will be used as center
points for the kernel calculation.
\end{description}

