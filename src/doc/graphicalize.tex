% This LaTeX document was generated using the LaTeX backend of PlDoc,
% The SWI-Prolog documentation system



\section{graphicalize.pl: Convert deductive databases into graphs}

\label{sec:graphicalize}

\begin{tags}
    \tag{author}
Paolo Frasconi
    \tag{To be done}
Safety checks, assumption checks, throwing exceptions, performance tuning
\end{tags}

This module contains predicates for converting a set of deductive
databases into a set of graphs, one for each interpretation, using
the mapping procedures described below.

\begin{description}
    \item[Concept] 
First the database is converted into a collection of ground facts by
computing intensional tables. Then the graphicalizer generates a
graph for each interpretation.
    \item[Database] 
The database is assumed to be in the following 'E/R' normal form:
\end{description}

\begin{itemize}
    \item Attributes are either object identifiers or properties.
    \item For each identifier i, there is at least one table t such that i
is the primary key of t. This table is basically the class or
entity-set of i.
    \item For each table, the primary key only consists of identifiers. So
multiway relationships are always represented by a table keyed by
the object identifiers involved in the relationship, augmented
with properties of the relationship.
\end{itemize}

\begin{description}
    \item[Graphicalization strategy] 
\end{description}

\begin{itemize}
    \item There is a vertex for each entity tuple (called an i-vertex), and
a vertex for each relationship tuple (an r-vertex). Vertices are
labeled by their tuples. Note that because of the above assumptions
there is one and only one i-vertex for each identifier.
    \item There is an edge between an i-vertex and an r-vertex if i belongs
to the tuple r. The graph is bipartite and there are no edges
between vertices of the same kind.
\end{itemize}

\begin{description}
    \item[Usage] 
The module provides two fundamental predicates: \predref{attach}{1}, for
loading a data set (variant \predref{attach}{0} for toy datasets asserted
directly in the kLog script), and \predref{make_graphs}{0}, \predref{make_graphs}{1} for
generating graphs.
    \item[History] 
git log
\end{description}

\vspace{0.7cm}

\begin{description}
    \predicate[det]{attach}{1}{+DataFile:atom_or_list_of_atoms}
Load a dataset i.e. a set of database instances (or interpretations)
by reconsulting the given file(s). Any previously attached data set
will be internally cleaned up so only one data set can be attached
at a time. Internally the predicate just consults the file
containing declarations of the predicate \qpredref{db}{interpretation}{2} (a
collection of ground facts). The predicate is stored in the
'graphicalize' module and should be never necessary to inspect it
from outside this module.

    \predicate[det]{attach}{0}{}
This first form is useful to attach a toy dataset written directly in
the kLog script. The data need to be declared as

\begin{code}
db:interpretation(Ex,Fact).
\end{code}

    \predicate[det]{detach}{0}{}
Retracts all data ground atoms and results of graphicalization.

    \predicate[det]{examples}{1}{-Set:list_of_atoms}
Returns in \arg{Set} the interpretation identifiers in the attached domain.

    \predicate[det]{database}{0}{}
For debugging purposes. List the extensional database.

    \predicate[det]{database}{1}{?Ex}
For debugging purposes. List the extensional database of interpretation \arg{Ex}.

    \predicate[det]{assert_background_knowledge}{1}{+Dataset:list_of_atoms}
Deduce intensional facts from the background knowledge and
interpretations listed in \arg{Dataset} and assert all of them (at the
top) in the form db:interpretation(Ex,Fact). Once asserted in this
way, intensional and extensional groundings are
indistinguishable. Use the first form to assert background knowledge
for every interpretation in the domain.

    \predicate[det,private]{save_view}{1}{+Filename}
Save a view of the data using only declared signatures. This allows
to write kLog scripts for data set transformation.

    \predicate[det]{make_graphs}{0}{}
Graphicalize the attached interpretations. Graphs are stored in
memory. 

    \predicate[det,private]{vertex_map}{3}{Ex:atom, SymID:atom, NumID:integer}
Dynamically asserted. Dictionary mapping a symbolic vertex
identifier (used in Prolog) to its numeric identifier in the
internal C++ representation of the graph. The symbolic identifier is
defined as follows. If v is an i-vertex, then \arg{SymID} is the tuple
identifier. If v is an r-vertex, then \arg{SymID} is the concatenation of
the signature name and the identifiers contained in it.

    \predicate[det,private]{add_graphs}{1}{+D:list_of_atoms}
For each interpretation id in \arg{D}, create a new empty C++ Graph object

    \predicate[det,private]{add_entity_vertices}{0}{}
Loop through active E-relations and add the corresponding vertices
to the internal collection of graphs (this is step 1 of graphicalization).

    \predicate[det,private]{add_relationship_vertices_and_edges}{0}{}
Loop through active R-relations and add the corresponding vertices
to the internal collection of graphs. Additionally, add edges to the
collection of graphs (this is step 2 of graphicalization).

    \predicate{map_prolog_ids_to_vertex_ids}{3}{+Ex:atom, +IDs:[atom], -NumericIds:[integer]}
Retrieve numeric vertex \arg{IDs} for a list of database identifiers \arg{IDs}
in interpretation \arg{Ex}

    \predicate[det]{domain_identifiers}{3}{?Ex, ?Type, ?Constants}
True if \arg{Constants} are the constants of type \arg{Type} in interpretation \arg{Ex}.

    \predicate[det,private]{search_identifiers}{0}{}
search identifiers from the ground entity sets. Results are asserted in
the database as \predref{domain_identifiers}{3}.

    \predicate[det]{check_db}{0}{}
Internal use. Performs some checks on the loaded data sets.
\end{description}

