% This LaTeX document was generated using the LaTeX backend of PlDoc,
% The SWI-Prolog documentation system



\section{doc/cinterface.pl: C++ interface}

\label{sec:cinterface}

General mechanism: functions listed here are the interface to
Prolog. All model-related predicates take an atom as first argument
that identifies the model (valid atoms should be strings, not
integers). The identifier is then used as a key for a dictionary
stored in the singleton The_ModelPool with associates the model
name to a pointer to an object of Model class (Model is the
abstract class from which all Models are inherited). Typically,
functions defined here should call a homologous method of the class
ModelPool that will dispatch the message to the actual model.

\subsection{Example}

Prolog code in the kLog script:

\begin{code}
new_model(my_model,libsvm_c_svc),
\end{code}

in c_interface, c_new_model() calls the method of the singleton The_ModelPool

\begin{code}
The_ModelPool.new_model("my_model","libsvm_c_svc")
\end{code}

which eventually creates an object of class
Libsvm_Model$<$BinaryClassifierReporter$>$. now back to the prolog
code, the atom my_model can be used as a handle for referring to
the C++ object just created, e.g.

\begin{code}
kfold(target_relation, 10, my_model, my_fg)
\end{code}

that causes the invocation of the method

\begin{code}
The_ModelPool.train("my_model",vector_of_interpretation_ids)
\end{code}

A similar mechanism is used for feature generators via the
singleton The_FeatureGeneratorPool of class FeatureGeneratorPool.

\subsection{Data}

Data is contained in the singleton The_Dataset of type
Dataset. There is a graph for each interpretation and possibly
several sparse vectors associated to the scalar predictions (called
cases in kLog). The former are accessed from prolog via
interpretation identifiers, the latter via "case" identifiers.

The_Dataset maintains internal dictionaries (indices) to
associate identifiers to graph pointers or sparse vector pointers.

Additionally, The_Dataset maintains a map from interpretation ids
to a set$<$string$>$ of case ids. Use The_Dataset.get_cases() to
retrieve this map.\vspace{0.7cm}

\begin{description}
    \predicate[det]{set_c_klog_flag}{3}{+Client:atom, +Flag:atom, +Value:atom}
Sets C-level kLog \arg{Flag} to \arg{Value} for \arg{Client}

    \predicate[det]{get_c_klog_flag}{3}{+Client, +Flag, -Value}
Unifies \arg{Value} with the current value of \arg{Flag} in \arg{Client}.

    \predicate[det]{document_klog_c_flags}{2}{+Client:atom, -Documentation:atom}
Retrieve \arg{Documentation} for all flags belonging to \arg{Client}.

    \predicate[nondet]{klog_c_flag_client}{1}{?Client}
Backtrack over the IDs of the C language flag clients.

    \predicate[nondet]{klog_c_flag_client_alt}{1}{?Client}
Backtrack over the IDs of the C language flag clients.
This implementation uses less memory, it doesn't store a copy of all alternatives in memory

    \predicate[det]{c_shasum}{2}{+Filename, -HashValue}
Unifies \arg{HashValue} with the SHA1 hash value of the file (which must exist).

    \predicate[det]{add_graph}{1}{+Id}
Create a new graph identified by \arg{Id}. \arg{Id} can be any valid Prolog
atom.

    \predicate[det]{add_graph_if_not_exists}{1}{+Id}
Create a new graph identified by \arg{Id} unless it already exists. \arg{Id}
can be any valid Prolog atom

    \predicate[det]{delete_graph}{1}{+Id}
Delete graph identified by \arg{Id}.

    \predicate[det]{write_graph}{1}{+Id}
Write graph identified by \arg{Id} on the standard output.

    \predicate{export_graph}{3}{+Id, +Filename, +Format}
Save graph identified by \arg{Id} into \arg{Filename} in specified
\arg{Format}. File extension is automatically created from format (it is
actually identical). Valid formats are dot, csv, gml, gdl, gspan.

    \predicate[det]{save_as_libsvm_file}{1}{+Filename}
Save a sparse vector dataset into \arg{Filename} in libsvm format. Data
is generated for all interpretations in the attached dataset. Every
line is a case. The line is ended by a comment giving the case
name, which is formed by concatenating the interpretation name and
the identifiers contained in the target fact.

    \predicate[det]{cleanup_data}{0}{}
Clean completely graphs and sparse vectors.

    \predicate[det]{clean_internals}{1}{+GId}
Clean up internal data structures on \arg{GId} to recover memory

    \predicate[det]{vertex_alive}{3}{+GId, +VId, ?State}
Set/get aliveness \arg{State} (yes/no) of vertex \arg{VId} in graph \arg{GId}

    \predicate[det]{set_signature_aliveness}{3}{+GId, +S, +State}
Set aliveness state to \arg{State} to every vertex of signature \arg{S} in graph \arg{GId}

    \predicate[det]{set_slice_aliveness}{3}{+GId, +Slice, +State}
Set aliveness state to Status to every vertex in slice \arg{Slice} in graph \arg{GId}

    \predicate[det]{set_signature_in_slice_aliveness}{4}{+GId, +S, +SliceID, +State}
Set aliveness state to \arg{State} to every vertex of signature \arg{S} in
slice \arg{SliceID} in graph \arg{GId}

    \predicate[det]{set_all_aliveness}{2}{+GId, +State}
Set aliveness state to \arg{State} in every vertex in graph \arg{GId}

    \predicate[det]{add_vertex}{8}{+GId, +SliceId, +Label, +IsKernelPoint, +IsAlive, +Kind, +SymId, -Id}
Create new vertex in graph \arg{GId} with label \arg{Label} and unify \arg{Id} with
the identifier of the new vertex Set KernelPoint to 'yes' to
indicate that this vertex is a kernel point Set \arg{IsAlive} to 'yes'
to indicate that this vertex is alive (i.e. actually exists). Dead
vertices are used for structured output predictions. \arg{Kind} should
be either 'r' for relationship or 'i' for an entity set. \arg{SymId} is
the symbolic id of the vertex and is only used for
visualization/debugging purposes. \arg{SliceId} is the slice to which
this vertex belongs to.

    \predicate[det]{add_edge}{5}{+GId, +V, +U, +Label, -Id}
Create new edge between \arg{U} and \arg{V} with label \arg{Label} and unify \arg{Id} with
the identifier of the new edge

    \predicate[det]{make_sparse_vector}{6}{+ModelType:atom, +FG:atom, +IntId:atom, +SliceId, +CaseId:atom, +ViewPoints:list_of_numbers}
Use feature generator \arg{FG} to make a sparse vector from graph Ex and
focusing on the list of vertices in \arg{ViewPoints}. The sparse vector
is then identified by \arg{CaseId}. \arg{ModelType} is used to construct an
appropriate internal representation of the feature vector.

    \predicate[det]{write_sparse_vector}{1}{+CaseId}
Write the sparse vector identified by \arg{CaseId}.

    \predicate[det]{add_feature_to_sparse_vector}{3}{+CaseId, +FeatureStr, +FeatureVal}
Adds new feature to sparse vector identified by \arg{CaseId}. \arg{FeatureStr}
is hashed to produce the index in the sparse vector. \arg{FeatureVal} is
then assigned to vector[hash(\arg{FeatureStr})]
This method was introduced for ghost signatures

    \predicate[det]{set_target_label}{2}{+CaseID, +Label}
Set the label for SVM \arg{CaseID}

    \predicate[det]{set_label_name}{2}{+Label:atom, +NumericLabel:number}
Record label name

    \predicate[det]{set_rejected}{1}{+CaseID}
Set reject flag for SVM \arg{CaseID}

    \predicate[det]{clear_rejected}{0}{}
Clear all reject flags

    \predicate[det]{remap_indices}{0}{}
Remap feature vector indices in a small range

    \predicate[det]{print_graph_ids}{0}{}
Print all IDs for graphs in the dataset

    \predicate[det]{new_feature_generator}{2}{+FGId, +FGType}
Create a new feature generator identified by atom \arg{FGId}

    \predicate[det]{delete_feature_generator}{1}{+Id}
Delete feature_generator identified by \arg{Id}.

    \predicate[det]{new_model}{2}{+ModelId, +ModelType}
Create a new model identified by atom \arg{ModelId}

    \predicate[det]{delete_model}{1}{+Id}
Delete model identified by \arg{Id}.

    \predicate[det]{model_type}{2}{+ModelId, -ModelType}
Unifies \arg{ModelType} with type of model identified by \arg{ModelId}.

    \predicate[det]{write_model}{1}{+Id}
Write model identified by \arg{Id}.

    \predicate[det]{load_model}{2}{+Id, +Filename}
Load model identified by \arg{Id} from \arg{Filename}

    \predicate[det]{save_model}{2}{+Id, +Filename}
Save model identified by \arg{Id} into \arg{Filename} in kLog internal format

    \predicate[det]{check_model_ability}{2}{+ModelId:atom, +Ability:atom}
Ask \arg{ModelId} whether it can perform the task specified in \arg{Ability}.

    \predicate[det]{train_model}{2}{+ModelId, +DatasetSpec}
Train model identified by ModelIdId. The list \arg{DatasetSpec} should
contain identifiers of interpretations. Case identifiers are then
generated in Pool.h

    \predicate[det]{test_dataset}{3}{+ModelId:atom, +DatasetSpec:atom, +NewData:atom}
Test model identified by ModelIdId on a dataset. The list
\arg{DatasetSpec} should contain identifiers of interpretations. Case
identifiers are then generated in Pool.h. If \arg{NewData} is "true" (or
"yes") then it is assumed that the evaluation is on test data and
results are accumulated in the global reporter of the
model. Otherwise it is assumed that we are testing on training data
and only the local reporter is affected.

    \predicate[det]{dot_product}{3}{+CaseId1, +CaseId2, -Value}
Unify \arg{Value} with the dot product of the feature vectors attached to
case identifiers \arg{CaseId1} and \arg{CaseId2}.

    \predicate[det]{set_model_wd}{2}{+ModelId, +WorkingDir}
Inform model of the working directory. This is mainly needed for
external models in order to keep a clean filesystem.

    \predicate[det]{reset_reporter}{2}{+ModelId, +Reporter}
Reset a reporter of the model. \arg{Reporter} should be either local or
global. 

    \predicate[det]{reset_reporter}{3}{+ModelId, +Reporter, +Description}
Reset a reporter of the model. \arg{Reporter} should be either local or
global. Also sets the description message to \arg{Description}.

    \predicate[det]{report}{2}{+ModelId, +Reporter}
Ask a performance reporter to write a report on the
standard output. \arg{Reporter} should be either local or global.

    \predicate[det]{report}{3}{+ModelId, +Filename, +Reporter}
Ask a performance reporter to write a report to \arg{Filename}. \arg{Reporter}
should be either local or global.

    \predicate[det]{save_predictions}{3}{+ModelId, +Dir, +Reporter}
Save predictions stored in a performance reporter in output.log
and maybe output.yyy in \arg{Dir}. Do AUC analysis for binary
classification reporters. \arg{Reporter} should be either local or global.

    \predicate[det]{get_prediction}{3}{+ModelId, +CaseID, -Margin}
Obtain prediction for \arg{CaseID} in the global reporter of model
identified by \arg{ModelId}. WARNING: fails without displaying any message
if arguments are not valid.
\end{description}

