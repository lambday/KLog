% This LaTeX document was generated using the LaTeX backend of PlDoc,
% The SWI-Prolog documentation system



\section{kfold.pl: k-fold cross-validation}

\label{sec:kfold}

\begin{tags}
    \tag{author}
Kurt De Grave, Paolo Frasconi
\end{tags}

This module contains predicates for estimating prediction accuracy
using k-fold cross validation or a random train/test split. A
``fold'' in this context refers to a set of interpretations, several
cases can be included in a given interpretation. Common testing
strategies like leave-one-university-out in WebKB or
leave-one-research-group-out in UW-CSE are naturally obtained by
setting k to the number of interpretations in the domain. The module
contains support for stratified k-fold (where strata are specified
by a user-defined predicate) and prepartitioned k-fold where folds
are read from file.

It is recommended that any empirical evaluation of kLog is run using
predicates in this module since they all save experimental results
on disk in a structured form (see below).

\begin{description}
    \item[Usage] 
See predicates \predref{kfold}{4}, \predref{random_split}{4}, \predref{fixed_split}{6},
\predref{prepartitioned_kfold}{4}, \predref{stratified_kfold}{5}.
    \item[Results] 
Results are saved in a structured way in the file system. \predref{kfold}{4}
and \predref{random_split}{4} create a results directory named after a SHA-1
hash of: (1) current klog flags, (2) domain traits, (3) train/test
specification, e.g.

\begin{code}
PWD/results/e789ba9abfa5e1ae77ec0f481dab9971f343df35
\end{code}

where PWD is the current working directory. This ensures that
different experiments run with different parameters (such as kernel
hyperparameters, regularization, etc) or background knowledge will
be kept separate and can coexist for subsequent analysis (e.g. model
selection).

For k-fold cross-valitation, rhe results directory is structured
into k sub-directories, one for each fold. These contain the
following files:

\begin{code}
auc.log
hash_key.log      text used to generate the hash key
output.log        predicted margins for each case in the form target margin # case term
output.yyy        same after stripping the case term
output.yyy.pr     recall-precision curve
output.yyy.roc    ROC curve
output.yyy.spr    precision points calculated at 100 recall points between 0 and 1.
test.txt          list of test interpretations in this fold
train.txt         list of training interpretations in this fold
\end{code}

@tbd Prediction on attributes (multiclass, regression)

@credit AUCCalculator by Jesse Davis and Mark Goadrich
\end{description}

\vspace{0.7cm}

\begin{description}
    \predicate[det]{stratified_kfold}{5}{+Signature, +K, +Models, +FeatureGenerator, +StratumFunctor}
Perform a stratified k-fold cross validation. The target \arg{Signature}
defines the learning task(s). \arg{K} is the numnber of folds. \arg{Models} is a
(list of) model(s) to be trained and \arg{FeatureGenerator} defines the
used kernel. \arg{StratumFunctor} is the name of a user-declared Prolog
predicate of the form stratum(+Interpretation,-Stratum); it should
unify Stratum (e.g. the category for classification) with the
stratum of the given Interpretation.

    \predicate[det]{prepartitioned_kfold}{4}{+Signature, +Models, +FeatureGenerator, +DefinitionFile}
Perform a repeated kfold cross-validation reading folds from a file
containing facts of the form test_int(Trial,Fold,IntId). Each trial
is saved into a separate subfolder (which in turns contains as many
folders as folds). This is useful for comparing against other
methods while keeping the same partitions, so that paired
statistical tests make sense.

    \predicate[det]{random_split}{4}{+Sig:atom, +Fraction:number, +Models:\{atom\}, +FGenerator:atom}
Train and test Model(s) on a given \arg{Fraction} of existing cases
(rest used for test). TrainFraction should be either a fraction in
(0.0,1.0) as a float or in (0,100) as an integer.

    \predicate[det]{fixed_split}{6}{+Sig:atom, +Comment:atom, +Train:[atom], +Test:[list], +Models:\{atom\}, +FGenerator:atom}
Do train/test on a given fixed split specified by \arg{Train} and
\arg{Test}. \arg{Comment} can be used to describe the split briefly (will go
into the results folder name).

    \predicate[det]{kfold}{4}{+Signature, +K, +Models, +FeatureGenerator}
Perform a k-fold cross validation. Identical to \predref{stratified_kfold}{5} except no stratum predicate is used.

    \predicate[private]{make_results_directory}{1}{-Dir:atom}
Creates a directory for results based on the SHA-1 code of current
klog flags, e.g. results/e789ba9abfa5e1ae77ec0f481dab9971f343df35,
unify \arg{Dir} with the created directory, and put a log of flags in
\arg{Dir}/flags.log
\end{description}

