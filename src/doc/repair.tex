% This LaTeX document was generated using the LaTeX backend of PlDoc,
% The SWI-Prolog documentation system



\section{repair.pl: Referential integrity constraints}

\label{sec:repair}

\begin{tags}
    \tag{See also}
\verb$klog_flag$ \verb$referential_integrity_repair$
\end{tags}

A referential integrity constraint violation occurs if an R-tuple
refers to an undefined identifier (i.e. there is no E-tuple identified
by this foreing key). Violations can be repaired or ignored.\vspace{0.7cm}

\begin{description}
    \predicate[det]{check_and_repair}{1}{+Ex:atom}
Scan the database for possible referential integrity violations and
repair them. The repair strategy is defined by the flag
\verb$referential_integrity_repair$. If the value is \const{add}, then the
defective E-tuple is added. If the value is \const{delete} then the
offending R-tuple is deleted. If the value is \const{ignore} an exception
is raised: simply don't call this predicate if no check should be
performed.
\end{description}

