\documentclass[a4,11pt]{memoir}

\setlength{\parindent}{0cm}
\setlength{\parskip}{0.15cm}

\setlrmarginsandblock{2.5cm}{*}{1} 
\setulmarginsandblock{2.5cm}{2.5cm}{*}
\setmarginnotes{2.5mm}{2cm}{1em}
\checkandfixthelayout

\usepackage[latin1]{inputenc}
\usepackage[english]{babel}
\usepackage[T1]{fontenc}
\usepackage{
  url,
  fancyvrb,
  multicol
}
\usepackage[colorlinks]{hyperref}
\usepackage{graphicx}
\usepackage[draft]{fixme}
\usepackage{fourier}
% \newcommand\starbreak{\fancybreak{\decosix\quad\decosix\quad\decosix}}

%\newcommand\starbreak{\fancybreak{$*\quad*\quad*$}}

\usepackage[scaled]{berasans}



% \usepackage[T1]{fontenc}
% \usepackage{tgpagella}
% \usepackage[sc]{mathpazo}
\usepackage{authblk}
% \usepackage{times}
\usepackage{pldoc2}


%% Margin notes
% \setlength{\marginparwidth}{3cm}
\let\oldmarginpar\marginpar
\renewcommand\marginpar[1]{\-\oldmarginpar[\raggedleft\footnotesize #1]%
{\raggedright\footnotesize \fbox{\textbf{#1}}}}
%%%%%%%%%%%


\sloppy
\makeindex

\title{kLog\\
  User's Manual}
\author[1,2]{Paolo Frasconi}
\author[2]{Fabrizio Costa}
\author[2]{Luc De Raedt}
\author[2]{Kurt De Grave}
\affil[1]{Dipartimento di Sistemi e Informatica, Universit{\`a} degli
  Studi di Firenze, via di Santa Marta 3, I-50139 Firenze, Italy}
\affil[2]{Departement Computerwetenschappen, Katholieke Universiteit
  Leuven, Celestijnenlaan 200A B-3001 Heverlee, Belgium}

% \setcounter{secnumdepth}{3}% to get numbered subsections
\setsecnumdepth{subsection}% to get numbered subsections

\begin{document}

\maketitle
\chapterstyle{demo3}

\frontmatter

\tableofcontents
% \setlength{\unitlength}{1pt}
% \clearpage
% \listoffigures
% \clearpage
% \listoftables
% \clearpage
% \listofegresults

\mainmatter
\chapterstyle{ell}


\chapter{Introduction}

This document provides basic information for kLog, a logical and
relational language for developing kernel-based learning systems.
kLog has been developed at K.U. Leuven and at Università degli Studi
di Firenze by P. Frasconi, F. Costa, L. De Raedt and K. De Grave.

License?

\chapter{Tutorial}
In this tutorial we explain by examples how to use kLog in several
relational learning data sets. You are encouraged to scan the whole
tutorial since most concepts are only explained the first time they
are used. 

kLog learns from interpretations, i.e. training data consist of sets
of ground atoms that are true in a given interpretation. An
alternative view is that each interpretation is an instance of a
relational database.

\section{Bursi}
This is a small molecules data set and the goal is the discrimination
between mutagenic and non-mutagenic compounds. Every molecule
corresponds to one interpretation so the kLog job is binary
classification of interpretations.

\marginpar{kLog data sets}
Data sets in kLog are stored as Prolog files consisting of collections
of ground facts. The predefined predicate \predref{interpretation}{2}
\index{interpretation/2} takes as its first argument an interpretation
identifier and as its second argument a ground term.

In the example below \predref{a,2} describes atoms, \predref{b,3}
describes chemical bonds, \predref{sub,3} functional groups,
\predref{subat,3} atom membership in functional groups,
\predref{target,1} describes the category of the molecule
(\texttt{mutagen} is the positive class, \texttt{nonmutagen} the
negative class), and \predref{linked,2} describes how functional
groups are linked together.

\begin{linumcode}
interpretation(i788,a(a1,c)).
interpretation(i788,a(a2,c)).
interpretation(i788,a(a3,cl)).
interpretation(i788,a(a4,br)).
interpretation(i788,a(a5,n)).
interpretation(i788,a(a6,h)).
interpretation(i788,b(a1,a2,1)).
interpretation(i788,b(a1,a3,1)).
interpretation(i788,b(a1,a4,1)).
interpretation(i788,b(a2,a5,3)).
interpretation(i788,b(a1,a6,1)).
interpretation(i788,sub(f1,halide,1)).
interpretation(i788,sub(f2,halide,1)).
interpretation(i788,sub(f3,nitrile,2)).
interpretation(i788,sub(f4,aliphatic_chain,1)).
interpretation(i788,subat(f1,a3,1)).
interpretation(i788,subat(f2,a4,1)).
interpretation(i788,subat(f3,a2,1)).
interpretation(i788,subat(f3,a5,2)).
interpretation(i788,subat(f4,a1,1)).
interpretation(i788,target(mutagen)).
interpretation(i788,linked(f4,[link(f3,a2,a1),link(f1,a3,a1),link(f2,a4,a1)],[branch(1,3)],saturated)).
\end{linumcode}

\begin{linumcode}
begin_domain.
signature atm(atom_id::self, element::property)::intensional.
atm(Atom, Element) :-
    a(Atom,Element), \+(Element=h).

signature bnd(atom_1@b::atm, atom_1@b::atm, type::property)::intensional.
bnd(Atom1,Atom2,Type) :-
    b(Atom1,Atom2,NType), describeBondType(NType,Type), atm(Atom1,_), atm(Atom2,_).

signature fgroup(fgroup_id::self, group_type::property)::intensional.
fgroup(Fg,Type) :- sub(Fg,Type,_).

signature fgmember(fg::fgroup, atom::atm)::intensional.
fgmember(Fg,Atom):- subat(Fg,Atom,_), atm(Atom,_).

signature fg_fused(fg1@nil::fgroup, fg2@nil::fgroup, nrAtoms::property)::intensional.
fg_fused(Fg1,Fg2,NrAtoms):- fus(Fg1,Fg2,NrAtoms,_AtomList).

signature fg_connected(fg1@nil::fgroup, fg2@nil::fgroup, bondType::property)::intensional.
fg_connected(Fg1,Fg2,BondType):-
    con(Fg1,Fg2,Type,_AtomList),describeBondType(Type,BondType).

signature fg_linked(fg::fgroup, alichain::fgroup, saturation::property)::intensional.
fg_linked(FG,AliChain,Sat) :-
    linked(AliChain,Links,_BranchesEnds,Saturation),
    (Saturation = saturated ->
     Sat = saturated
    ;
     Sat = unsaturated
    ),
    member(link(FG,_A1,_A2),Links).

signature mutagenic::intensional.
mutagenic :- target(mutagen).
end_domain.
\end{linumcode}

\section{Biodeg}
\section{UW-CSE}
\section{WebKB}
\section{IMDB}

% The rest of this LaTeX document is generated using the LaTeX
% backend of PlDoc, The SWI-Prolog documentation system

\chapter{Main components of the kLog Prolog library}
% This LaTeX document was generated using the LaTeX backend of PlDoc,
% The SWI-Prolog documentation system



\section{syntax.pl: kLog syntax}

\label{sec:syntax}

\begin{tags}
    \tag{author}
Paolo Frasconi
    \tag{To be done}
Safety checks, assumption checks, performance tuning
\end{tags}

This module provides syntactic extensions to Prolog for declaring
signatures. Most of the functionalities in this module are
implemented via \predref{term_expansion}{2}.

\begin{description}
    \item[History] 
see git log
\end{description}

\vspace{0.7cm}

\begin{description}
    \prefixop{signature}{+S}
Declare \arg{S} to be the type signature of a table, according to the following syntax:

\begin{code}
   <signature> ::= <header> [<sig_clauses>]
   <sig_clauses> ::= <sig_clause> | [<sig_clause> <sig_clauses>]
   <sig_clause> ::= <Prolog_clause>
   <header> ::= <sig_name> "(" <args> ")" "::" <level> "."
   <sig_name> ::= <Prolog_atom>
   <args> ::= <arg> | [<arg> <args>]
   <arg> ::= <column_name> [<role_overrider>] "::" <type>
   <column_name> ::= <Prolog_atom>
   <role_overrider> ::= "@" <role>
   <role> ::= <Prolog_atom>
   <type> ::= "self" | <sig_name>
   <level> ::= "intensional" | "extensional"
\end{code}

Arguments are given by default a role that corresponds to their
position in the argument list. The role can be overridden using the
@ syntax, e.g. to represent undirected relations. By default, the
role of the i-th argument is: * functor+i if the argument is an
identifier * functor+t if the argument is a property and t its type

thus, e.g. if a_id and b_id are identifier names, and c_type is a
property name, the following signatures are equivalent:

\begin{code}
signature foo(g_id::gnus,t_id::tnus,c::property)::extensional.
signature foo(g_id@1,t_id@2,c@3::property)::extensional.
\end{code}

Forcing the role can also easily implement undirected edges. Here is an
example from mutagenesis:

\begin{code}
signature atm( atom_id::self,
               element::property,
               quantaval::property,
               chargeval::property).
signature bond( atom_id@b::atom,
                atom_id@b::atom,
                bondtype::property).
\end{code}

More complex cases can be easily conceived, e.g. one can assign the
same role to every atom in a benzene ring but distinguish different
rings in a functional group like ball3 or phenanthrene. See
experiments/mutagenesis for details.

    \predicate[det,private]{expand_signature}{2}{+S, +Level}
Auxiliary predicate for expanding a signature declaration where \arg{S} is
a fact with typed arguments and \arg{Level} is either 'intensional' or
'extensional'. This predicate performs a number of checks and
asserts various signature traits (see below).
\predref{term_expansion}{2}.

    \predicate[det]{domain_traits}{2}{?TraitsSpec, ?TraitsVal}
Query domain traits. \arg{TraitsSpec} is one of:

\begin{itemize}
    \item signatures: list of all signature names
    \item entities: list of all entity signature names
    \item relationships: list of all relationship signature names
\end{itemize}

    \predicate[det]{domain_traits}{0}{}
List on the standard output all the domain traits and signature
traits for all signatures.

    \predicate[det]{domain_traits}{1}{+Stream}
List on output \arg{Stream} all the domain traits and signature
traits for all signatures.

    \predicate[det]{signature_traits}{3}{?Name, ?TraitsSpec, ?TraitsVal}
Query signature traits. \arg{Name} is the signature name. \arg{TraitsSpec} is one of:

\begin{itemize}
    \item kind: either entity or relationship
    \item arity: the number of arguments
    \item level: either intensional (deduced) or extensional
    \item column_types: list of argument types
    \item column_names: list of argument names
    \item column_roles: list of argument roles
    \item relational_arity: the number of non-property arguments
    \item ext_clause: goal for finding all ground tuples for this signature
The value is returned in \arg{TraitsVal}.
\end{itemize}

    \predicate[semidet]{id_position}{2}{+S, -Position}
If \arg{S} is the name of a valid entity signature, unify \arg{Position} with
the position of the identifier in the argument list. Otherwise fail.

    \predicate[semidet]{extract_properties}{3}{+Functor, +Args, -Properties}
If \arg{Functor} is the name of a valid signature whose arity matches the
length of the list \arg{Args}, then unify \arg{Properties} with the list of
items in \arg{Args} that appear at positions of property type. Otherwise
fail.

    \predicate[semidet]{extract_identifiers}{3}{+Functor, +Args, -IDs}
If \arg{Functor} is the name of a valid signature whose arity matches the
length of the list \arg{Args}, then unify \arg{IDs} with the list of items in
\arg{Args} that appear at positions of identifier type. Otherwise fail.

    \predicate[semidet]{extract_references}{2}{+S, -Refs}
If s is the name of a valid signature, then unify \arg{Refs} with the
subsequence of entity types referenced by \arg{S} but $>$self$<$ is replaced
by \arg{S}. If a type is referred to multiple times, it appears multiple
time in \arg{Refs}.

    \predicate[semidet]{is_kernel_point}{2}{+Functor, -YesNo}
If \arg{Functor} is the name of a valid signature then unify \arg{YesNo} with
'yes' iff the signature has declared a neighborhood method. All
graph vertices expanded from this signature will be used as center
points for the kernel calculation.
\end{description}


% This LaTeX document was generated using the LaTeX backend of PlDoc,
% The SWI-Prolog documentation system



\section{db.pl: kLog database interface}

\label{sec:db}

\begin{tags}
    \tag{author}
Paolo Frasconi
    \tag{To be done}
Maybe an interface to a DBMS such as MySQL
\end{tags}

This module provides access to the extensional database. Currently
it just consults a Prolog database ensuring that facts are loaded
into the db namespace.

\begin{description}
    \item[Syntactical conventions] 
The extensional declarations should use the special predicate
\qpredref{db}{interpretation}{2} whose first argument is an identifier for the
interpration (any Prolog term) and whose second argument is a fact.
    \item[History] 
see git log
\end{description}

\vspace{0.7cm}

\begin{description}
    \predicate[det]{db_consult}{1}{+File_s:atom_or_list}
Consult a collection of ground facts from File(s). Also accepts
basenames of .pl.gz gzipped files.
\end{description}


% This LaTeX document was generated using the LaTeX backend of PlDoc,
% The SWI-Prolog documentation system



\section{graphicalize.pl: Convert deductive databases into graphs}

\label{sec:graphicalize}

\begin{tags}
    \tag{author}
Paolo Frasconi
    \tag{To be done}
Safety checks, assumption checks, throwing exceptions, performance tuning
\end{tags}

This module contains predicates for converting a set of deductive
databases into a set of graphs, one for each interpretation, using
the mapping procedures described below.

\begin{description}
    \item[Concept] 
First the database is converted into a collection of ground facts by
computing intensional tables. Then the graphicalizer generates a
graph for each interpretation.
    \item[Database] 
The database is assumed to be in the following 'E/R' normal form:
\end{description}

\begin{itemize}
    \item Attributes are either object identifiers or properties.
    \item For each identifier i, there is at least one table t such that i
is the primary key of t. This table is basically the class or
entity-set of i.
    \item For each table, the primary key only consists of identifiers. So
multiway relationships are always represented by a table keyed by
the object identifiers involved in the relationship, augmented
with properties of the relationship.
\end{itemize}

\begin{description}
    \item[Graphicalization strategy] 
\end{description}

\begin{itemize}
    \item There is a vertex for each entity tuple (called an i-vertex), and
a vertex for each relationship tuple (an r-vertex). Vertices are
labeled by their tuples. Note that because of the above assumptions
there is one and only one i-vertex for each identifier.
    \item There is an edge between an i-vertex and an r-vertex if i belongs
to the tuple r. The graph is bipartite and there are no edges
between vertices of the same kind.
\end{itemize}

\begin{description}
    \item[Usage] 
The module provides two fundamental predicates: \predref{attach}{1}, for
loading a data set (variant \predref{attach}{0} for toy datasets asserted
directly in the kLog script), and \predref{make_graphs}{0}, \predref{make_graphs}{1} for
generating graphs.
    \item[History] 
git log
\end{description}

\vspace{0.7cm}

\begin{description}
    \predicate[det]{attach}{1}{+DataFile:atom_or_list_of_atoms}
Load a dataset i.e. a set of database instances (or interpretations)
by reconsulting the given file(s). Any previously attached data set
will be internally cleaned up so only one data set can be attached
at a time. Internally the predicate just consults the file
containing declarations of the predicate \qpredref{db}{interpretation}{2} (a
collection of ground facts). The predicate is stored in the
'graphicalize' module and should be never necessary to inspect it
from outside this module.

    \predicate[det]{attach}{0}{}
This first form is useful to attach a toy dataset written directly in
the kLog script. The data need to be declared as

\begin{code}
db:interpretation(Ex,Fact).
\end{code}

    \predicate[det]{detach}{0}{}
Retracts all data ground atoms and results of graphicalization.

    \predicate[det]{examples}{1}{-Set:list_of_atoms}
Returns in \arg{Set} the interpretation identifiers in the attached domain.

    \predicate[det]{database}{0}{}
For debugging purposes. List the extensional database.

    \predicate[det]{database}{1}{?Ex}
For debugging purposes. List the extensional database of interpretation \arg{Ex}.

    \predicate[det]{assert_background_knowledge}{1}{+Dataset:list_of_atoms}
Deduce intensional facts from the background knowledge and
interpretations listed in \arg{Dataset} and assert all of them (at the
top) in the form db:interpretation(Ex,Fact). Once asserted in this
way, intensional and extensional groundings are
indistinguishable. Use the first form to assert background knowledge
for every interpretation in the domain.

    \predicate[det,private]{save_view}{1}{+Filename}
Save a view of the data using only declared signatures. This allows
to write kLog scripts for data set transformation.

    \predicate[det]{make_graphs}{0}{}
Graphicalize the attached interpretations. Graphs are stored in
memory. 

    \predicate[det,private]{vertex_map}{3}{Ex:atom, SymID:atom, NumID:integer}
Dynamically asserted. Dictionary mapping a symbolic vertex
identifier (used in Prolog) to its numeric identifier in the
internal C++ representation of the graph. The symbolic identifier is
defined as follows. If v is an i-vertex, then \arg{SymID} is the tuple
identifier. If v is an r-vertex, then \arg{SymID} is the concatenation of
the signature name and the identifiers contained in it.

    \predicate[det,private]{add_graphs}{1}{+D:list_of_atoms}
For each interpretation id in \arg{D}, create a new empty C++ Graph object

    \predicate[det,private]{add_entity_vertices}{0}{}
Loop through active E-relations and add the corresponding vertices
to the internal collection of graphs (this is step 1 of graphicalization).

    \predicate[det,private]{add_relationship_vertices_and_edges}{0}{}
Loop through active R-relations and add the corresponding vertices
to the internal collection of graphs. Additionally, add edges to the
collection of graphs (this is step 2 of graphicalization).

    \predicate{map_prolog_ids_to_vertex_ids}{3}{+Ex:atom, +IDs:[atom], -NumericIds:[integer]}
Retrieve numeric vertex \arg{IDs} for a list of database identifiers \arg{IDs}
in interpretation \arg{Ex}

    \predicate[det]{domain_identifiers}{3}{?Ex, ?Type, ?Constants}
True if \arg{Constants} are the constants of type \arg{Type} in interpretation \arg{Ex}.

    \predicate[det,private]{search_identifiers}{0}{}
search identifiers from the ground entity sets. Results are asserted in
the database as \predref{domain_identifiers}{3}.

    \predicate[det]{check_db}{0}{}
Internal use. Performs some checks on the loaded data sets.
\end{description}


% This LaTeX document was generated using the LaTeX backend of PlDoc,
% The SWI-Prolog documentation system



\section{kfold.pl: k-fold cross-validation}

\label{sec:kfold}

\begin{tags}
    \tag{author}
Kurt De Grave, Paolo Frasconi
\end{tags}

This module contains predicates for estimating prediction accuracy
using k-fold cross validation or a random train/test split. A
``fold'' in this context refers to a set of interpretations, several
cases can be included in a given interpretation. Common testing
strategies like leave-one-university-out in WebKB or
leave-one-research-group-out in UW-CSE are naturally obtained by
setting k to the number of interpretations in the domain. The module
contains support for stratified k-fold (where strata are specified
by a user-defined predicate) and prepartitioned k-fold where folds
are read from file.

It is recommended that any empirical evaluation of kLog is run using
predicates in this module since they all save experimental results
on disk in a structured form (see below).

\begin{description}
    \item[Usage] 
See predicates \predref{kfold}{4}, \predref{random_split}{4}, \predref{fixed_split}{6},
\predref{prepartitioned_kfold}{4}, \predref{stratified_kfold}{5}.
    \item[Results] 
Results are saved in a structured way in the file system. \predref{kfold}{4}
and \predref{random_split}{4} create a results directory named after a SHA-1
hash of: (1) current klog flags, (2) domain traits, (3) train/test
specification, e.g.

\begin{code}
PWD/results/e789ba9abfa5e1ae77ec0f481dab9971f343df35
\end{code}

where PWD is the current working directory. This ensures that
different experiments run with different parameters (such as kernel
hyperparameters, regularization, etc) or background knowledge will
be kept separate and can coexist for subsequent analysis (e.g. model
selection).

For k-fold cross-valitation, rhe results directory is structured
into k sub-directories, one for each fold. These contain the
following files:

\begin{code}
auc.log
hash_key.log      text used to generate the hash key
output.log        predicted margins for each case in the form target margin # case term
output.yyy        same after stripping the case term
output.yyy.pr     recall-precision curve
output.yyy.roc    ROC curve
output.yyy.spr    precision points calculated at 100 recall points between 0 and 1.
test.txt          list of test interpretations in this fold
train.txt         list of training interpretations in this fold
\end{code}

@tbd Prediction on attributes (multiclass, regression)

@credit AUCCalculator by Jesse Davis and Mark Goadrich
\end{description}

\vspace{0.7cm}

\begin{description}
    \predicate[det]{stratified_kfold}{5}{+Signature, +K, +Models, +FeatureGenerator, +StratumFunctor}
Perform a stratified k-fold cross validation. The target \arg{Signature}
defines the learning task(s). \arg{K} is the numnber of folds. \arg{Models} is a
(list of) model(s) to be trained and \arg{FeatureGenerator} defines the
used kernel. \arg{StratumFunctor} is the name of a user-declared Prolog
predicate of the form stratum(+Interpretation,-Stratum); it should
unify Stratum (e.g. the category for classification) with the
stratum of the given Interpretation.

    \predicate[det]{prepartitioned_kfold}{4}{+Signature, +Models, +FeatureGenerator, +DefinitionFile}
Perform a repeated kfold cross-validation reading folds from a file
containing facts of the form test_int(Trial,Fold,IntId). Each trial
is saved into a separate subfolder (which in turns contains as many
folders as folds). This is useful for comparing against other
methods while keeping the same partitions, so that paired
statistical tests make sense.

    \predicate[det]{random_split}{4}{+Sig:atom, +Fraction:number, +Models:\{atom\}, +FGenerator:atom}
Train and test Model(s) on a given \arg{Fraction} of existing cases
(rest used for test). TrainFraction should be either a fraction in
(0.0,1.0) as a float or in (0,100) as an integer.

    \predicate[det]{fixed_split}{6}{+Sig:atom, +Comment:atom, +Train:[atom], +Test:[list], +Models:\{atom\}, +FGenerator:atom}
Do train/test on a given fixed split specified by \arg{Train} and
\arg{Test}. \arg{Comment} can be used to describe the split briefly (will go
into the results folder name).

    \predicate[det]{kfold}{4}{+Signature, +K, +Models, +FeatureGenerator}
Perform a k-fold cross validation. Identical to \predref{stratified_kfold}{5} except no stratum predicate is used.

    \predicate[private]{make_results_directory}{1}{-Dir:atom}
Creates a directory for results based on the SHA-1 code of current
klog flags, e.g. results/e789ba9abfa5e1ae77ec0f481dab9971f343df35,
unify \arg{Dir} with the created directory, and put a log of flags in
\arg{Dir}/flags.log
\end{description}


% This LaTeX document was generated using the LaTeX backend of PlDoc,
% The SWI-Prolog documentation system



\section{learn.pl: Learning support in kLog}

\label{sec:learn}

\begin{tags}
    \tag{author}
Paolo Frasconi
\end{tags}

Formulate kLog learning jobs. Below are some examples of target
relations that kLog may be asked to learn. Currently, implemented
models have abilities to cover only a fraction of these.

\subsection{Jobs with relational arity = 0:}

\subsubsection{Binary classification of each interpretation.}

\textbf{Example}: classification of small molecules.

\begin{code}
signature signature mutagenic::extensional.
\end{code}

\subsubsection{Regression for each interpretation.}

\textbf{Example}: predict the affinity of small molecules binding.

\begin{code}
signature affinity(strength::property)::extensional.
\end{code}

\subsubsection{Multitask on individual interpretations.}

\textbf{Example}: predict mutagenicity level and logP for small molecules.

\begin{code}
signature
  molecule_properties(mutagenicity::property(real),
                     logp::property(real))::extensional.
\end{code}

\subsubsection{Multiclass classification for each interpretation}

\textbf{Example}: image categorization;

\begin{code}
signature image_category(cat::property)::extensional.
\end{code}

\subsection{Jobs with relational arity = 1:}

\subsubsection{Binary classification of entities in each interpretation}

\textbf{Examples}: detect spam webpages as in the Web Spam Challenge
(\url{http://webspam.lip6.fr/wiki/pmwiki.php),} predict blockbuster
movies in IMDb

\begin{code}
signature spam(url::page)::extensional.
signature blockbuster(m::movie)::extensional.
\end{code}

\subsubsection{Multiclass classification of entities in each interpretation}

\textbf{Examples}: WebKB, POS-Tagging, NER, protein secondary structure prediction

\begin{code}
signature page(url::page,category::property)::extensional.
signature pos_tag(word::position,tag::property)::extensional.
signature named_entity(word::position,ne::property)::extensional.
signature secondary_structure(r::residue,ss::property)::extensional.
\end{code}

\subsubsection{Multiclass classification of entities in each interpretation}

\textbf{Example}: traffic flow forecasting

\begin{code}
signature flow_value(s::station,flow::property(real))::extensional.
\end{code}

\subsection{Jobs with relational arity = 2:}

\subsubsection{Link prediction tasks}

\textbf{Examples}: protein beta partners, UW-CSE, Entity resolution,
Protein-protein interactions

\begin{code}
signature partners(r1::residue,r2::residue)::extensional.
signature advised_by(p1::person,p2::person)::extensional.
signature same_venue(v1::venue,v2::venue)::extensional.
signature phosphorylates(p1::kinase,p2::protein)::extensional.
signature regulates(g1::gene,g2::gene)::extensional.
\end{code}

\subsubsection{Regression on pairs of entities}

\textbf{Example}: traffic flow forecasting at different stations and
different lead times, Prediction of distance between protein
secondary structure elements

\begin{code}
signature congestion(s::station,lead::time,
                     flow::property(float))::extensional.
signature distance(sse1:sse,sse2:sse,d::property(float))::intensional.
\end{code}

\subsubsection{Pairwise hierarchical classification}

\textbf{Example}: traffic congestion level forecasting at different stations and different lead times

\begin{code}
signature congestion_level(s::station,lead::time,
                           level::property)::extensional.
\end{code}

\subsection{Tasks with relational arity $>$ 2:}

\subsubsection{Prediction of hyperedges}

\textbf{Example}: Spatial role labeling

\begin{code}
signature ttarget(w1::sp_ind_can, w2::trajector_can,
                  w3::landmark_can)::intensional.
\end{code}

\textbf{Example}: Metal binding geometry

\begin{code}
signature binding_site(r1::residue,r2::residue,
                       r3::residue,r4::residue)::extensional.
\end{code}

\subsubsection{Classification of hyperedges}

\textbf{Example}: Metal binding geometry with ligand prediction

\begin{code}
signature binding_site(r1::residue,r2::residue,
                       r3::residue,r4::residue,
                       metal::property)::extensional.
\end{code}

\subsection{Subsampling}

The following predefined dynamic predicate fails by default:

\begin{code}
  user:klog_reject(+Case,+TrainOrPredict)
\end{code}

By defining your own version of it, it is easy to implement
selective subsampling of cases. A typical usage would be for dealing
with a highly imbalanced data set where you want to subsample only a
fraction of negatives. Case is a Prolog callable predicate
associated with a training or testing case. When kLog generates
cases, it calls \qpredref{user}{klog_reject}{2} to determine if the case should
be rejected. TrainOrPredict should be either 'train' or 'predict' to
do subsampling either at training or testing time. For example
suppose we have a binary classification task and we want to reject
90 percent of the negatives at training time. Then the following
code should be added to the kLog script.

\begin{code}
klog_reject(Case,train) :-
  \+ call(Case),
  random(R),
  R>0.1.
\end{code}

\subsection{Predicates}

\vspace{0.7cm}

\begin{description}
    \predicate[det,private]{depends_transitive}{2}{?S1:atom, ?S:atom}
Tabled. Contains dependency analysis results to properly kill vertices in
the graphs. The predicate succeeds if \arg{S1} ``depends'' on \arg{S}. The
mechanism actually goes a bit beyond a simple transitive closure of
the call graph: when several targets are present in the same file it
is necessary to extend the definition of dependencies. The set of
dependent signatures is defined as follows:

\begin{enumerate}
    \item All ancestors, including \arg{S} itself (obvious)
    \item All the descendants (less obvious; however, if a target signature
\arg{S} is intensional and calls some predicate Y in its definition, then
Y supposedly (although not necessarily) contains some supervision
information, which should be removed).
    \item All the ancestors of the descendants (maybe even less obvious,
but if now there is some other signature T which calls Y, then T
will also contain supervision information).

The descendants of the ancestors need not to be removed.

Some of the ancestors of \arg{S} might be legitimate predicates that have
nothing to do with supervision. Therefore we restrict ourselves to
the subset of the call graph with predicates that are either in the
user: namespace (where the 'user' could cheat accessing supervision
information) or in the data set file (in the db: namespace). In any
case, since removal of vertices from the graph might surprise the
user, kLog issues a warning if there are killed signatures besides
the target signature of the current training or test
procedure. Finally, one can declare a signature to be safe using
\predref{safe}{1}. In this case it is assumed that the Prolog code inside it
does not exploit any supervision information. \predref{safe}{1} should be used
with care.

The table depends on \predref{current_depends_directly}{2} which is asserted by
\predref{record_dependencies}{0}, called at the end of \qpredref{graphicalize}{attach}{1}.
\end{enumerate}

    \predicate[det]{train}{4}{+S:atom, +Examples:list_of_atoms, +Model:atom, +Feature_Generator:atom}
Train \arg{Model} on a data set of interpretations, using
\arg{Feature_Generator} as the feature generator. \arg{Examples} is a list of
interpretation identifiers. \arg{S} is a signature. For each possible
combination of identifiers in \arg{S}, a set of "cases" is generated where
a case is just a pseudo-iid example. If \arg{S} has no properties,
positive cases are those for which a tuple exist in the
interpretation and negative cases all the rest. If \arg{S} has one
property, this property acts as a label for the case (can be also
regression if the property is a real number). Two or more properties
define a multi-task problem (currently handled as a set of
independent tasks). Problems like small molecules classification
have a target signature with zero arity which works fine as a
special case of relationship. \arg{Model} should have the ability of
solving the task specified in the target signature \arg{S}.

\begin{tags}
\mtag{Errors}- if the target signature is a kernel point (training on it would be cheating) \\- if \arg{Model} cannot solve the task specified by \arg{S}.
    \tag{To be done}
Multitask taking correltations into account
\end{tags}

    \predicate[det,private]{kill_present}{2}{+TargetSignature:atom, +Examples:list}
Predicates \predref{kill_present}{2} and \predref{kill_future}{1} are needed to setup training and
test data. Let's
call vertices associated with the target signature plus all their
dependents (defined by \predref{depends_transitive}{2}) the set of "query"
vertices. In the case of unsliced interpretations, all query
vertices must be killed. The case of sliced interpretations is more
tricky and is handled as follows:

Example using IMdB, slice_preceq defined by time (years)

Suppose data set contains imdb(1953),...,imdb(1997) and that we want
to train on [imdb(1992),imdb(1993)] and test on
[imdb(1995),imdb(1996)]. Then during training we first take the most
recent year in the training set (1993) and kill \textbf{everything}
strictly in the future (i.e. 1994, 1995, 1996, 1997) using
\predref{kill_future}{1} plus the query vertices in the present (i.e. 1992 and
1993) using \predref{kill_present}{2}. Test is similar, we take the most recent
year in the test set (1996) and kill \textbf{everything} strictly in the
future (i.e. 1997) plus the query vertices in the present (i.e. 1995
and 1996). Thus, for example, movies of 1994 are not used for
training but during prediction their labels are (rightfully)
accessible for computing feature vectors. In a hypothetical
transductive setting we might keep alive vertices in the future for
"evidence" signatures. However this is not currently supported.

\predref{kill_present}{2} kills vertices associated with the \arg{TargetSignature}
(and dependents) in the present slices. If interpretations are not
sliced vertices are killed anyway.

    \predicate[det,private]{kill_future}{1}{+Examples:list}
Kill entire slices in the strict future (if there are no sliced
interpretations \predref{kill_future}{1} does nothing since max_slice will
fail)

    \predicate[det,private]{list_of_slices}{2}{+SlicedInterpretations, ?Slices}
\arg{Slices} is unified with the list of slices found in \arg{SlicedInterpretations}.

    \predicate[det,private]{preceq_max_list}{2}{+List, ?M}
\arg{M} is unified with the max element in \arg{List} according to the total
order defined by \qpredref{db}{slice_preceq}{2}.

    \predicate[det,private]{preceq_min_list}{2}{+List, ?M}
\arg{M} is unified with the min element in \arg{List} according to the total
order defined by \qpredref{db}{slice_preceq}{2}.

    \predicate[det]{predict}{4}{+S:atom, +Examples:list_of_atoms, +Model:atom, +Feature_Generator:atom}
Test \arg{Model} on a data set of interpretations. Use \arg{Feature_Generator} as the feature
generator. \arg{Examples} is a list of interpretation identifiers. \arg{S} is a
signature. Cases are generated as in \predref{train}{3}. The predicate asserts
induced facts as follows:

\begin{code}
induced(InterpretationId,db:interpretation(InterpretationID,Fact)).
\end{code}

    \predicate[nondet,private]{get_task}{4}{+TargetSignature:atom, -TaskIndex:integer, -TaskName:atom, -Values:list}
Given \arg{TargetSignature}, retrieve the i-th task (starting from 0),
unify \arg{TaskIndex} with i, \arg{TaskName} with its name and \arg{Values} with the
list of target values found in the data set for this task. \arg{TaskName}
is either in the form N\#P where N is the property name and P the
position in the argument list, or the atom 'callme' meaning that the
task is to learn a relationship.

    \predicate[det,private]{cases_loop}{7}{+TS, +Examples, +Model, +FG, +TName, +TType, +TOrP}
Core procedure for training (if \arg{TOrP}=train) or testing (if
\arg{TOrP}=predict). \arg{TS} is the target signature. \arg{Examples} is a list of
interpretations (from which cases are built). \arg{Model} is a kLog
model. \arg{FG} is a kLog feature generator. \arg{TName} and \arg{TType} are the task
name and type as returned by \predref{get_task}{4}. The loop generates all
cases for each interpretation (using \predref{tuple_of_identifiers}{3}) and
creates (if necessary) all required feature vectors and output
labels. In prediction mode, cases are immediately predicted. In
training mode, C++-level predicates \predref{train_model}{2} and \predref{test_dataset}{3}
are called at the end of the loop. In both cases, results are
accumulated both in the local and the global reporters. However the
local reporter is reset when this loop starts. This is useful for
obtaining training set accuracy and test set accuracy of individual
folds in k-fold-CV.

    \predicate[nondet,private]{tuple_of_identifiers}{3}{+Ex, +S, -List}
Unify IDList with a tuple of identifiers in \arg{Ex} whose type appears in
signature \arg{S}. On backtracking will retrieve all possible tuples. For
tasks such as link prediction, this predicate is used to generate
all pairs of candidates.

    \predicate[nondet,private]{identifier}{3}{+Ex, +S, -ID}
Unify \arg{ID} with one of the identifiers of \arg{S} in \arg{Ex}. On backtracking
will return all data identifiers for signature \arg{S} in interpretation
\arg{Ex}. The predicate fails if \arg{S} is not an entity.

    \predicate[private]{clean_internals}{1}{IntId}

    \predicate[det,private]{prolog_make_sparse_vector}{5}{Model, Feature_Generator, Ex, CaseID, ViewPoint}
Wrapper around C++ method for making feature vectors.
\arg{Feature_Generator} is the name of the feature generator object that
will be used. \arg{Ex} is the interpretation name, possibly sliced. \arg{CaseID}
identifies the case for which the feature vector is
generated. \arg{ViewPoint} is a list of vertex IDs (C++ integer code)
around which the feature vector is constructed. For unsliced
interpretations, the feature vector is registered under a
slash-separated string like ai/advised_by/person20/person240,
created from the interpretation identifier (e.g. ai) followed by the
target signature (e.g. advised_by) and the identifiers of the
entities that define the case, (e.g. person20 and person412). For
sliced interpretations the slice name is also used to construct the
identifiers, e.g. imdb_1997 and imdb_1997/m441332. \arg{Model} is used to
determine the internal format of the feature vector being generated.

    \predicate[det,private]{make_case_id}{4}{Ex, S, IDTuple, CaseID}
Unifies \arg{CaseID} with a unique identifier for (sliced) interpretation
\arg{Ex}, target signature \arg{S}, and tuple of identifiers \arg{IDTuple}. See
\predref{prolog_make_sparse_vector}{5} for details on how this is formatted.

    \predicate[det]{save_induced_facts}{2}{Signatures, Filename}
Every 'induced' fact (asserted during test) is saved to \arg{Filename} for
a rough implementation of iterative relabeling. The target signature
is renamed by prefixing it with 'pred_'.
\end{description}


% This LaTeX document was generated using the LaTeX backend of PlDoc,
% The SWI-Prolog documentation system



\section{flags.pl: flags}

\label{sec:flags}

\begin{tags}
    \tag{author}
Paolo Frasconi
\end{tags}

Module for declaring and manipulating klog flags. For a list of
supported flags use \predref{klog_flags}{2}.

\begin{description}
    \item[Developer note] To add more flags, use \predref{flag_traits}{4} as

\begin{code}
flag_traits(FlagName,C_Or_Prolog,Checker,Description)
\end{code}

where FlagName is an atom naming the flag, C_Or_Prolog is either the
atom c or the atom prolog, Checker is a predicate that checks
whether a value for this flag is legal, and Description is an atom
describing the flag and its legal values. Prolog flags are directly
handled in Prolog. C flags are used by the underlying kernel
calculation system in C++. The declaration mechanism and default
value assignment is slightly different in the two cases.
\end{description}

\vspace{0.7cm}

\begin{description}
    \predicate[det]{klog_flag}{2}{+FlagName:atom, ?Val:atom}
If \arg{Val} is a free variable, unify it with the current value of the
flag \arg{FlagName}. Otherwise sets \arg{FlagName} to \arg{Val}.

    \predicate[semidet]{klog_flag}{3}{+Who:atom, +FlagName:atom, ?Val:atom}
If \arg{Val} is a free variable, unify it with the current value of the
flag \arg{FlagName}. Otherwise sets \arg{FlagName} to \arg{Val}. \arg{Who} is the handle of
the kLog object for which the flag is read or set.

    \predicate[det]{set_klog_flag}{2}{+FlagName:atom, +Val:atom}
Sets \arg{FlagName} to \arg{Val}.

    \predicate[det]{set_klog_flag}{3}{+Who:atom, +FlagName:atom, +Val:atom}
Sets \arg{FlagName} to \arg{Val} for C++ object \arg{Who}.

    \predicate[semidet]{get_klog_flag}{2}{+FlagName:atom, -Val}
Unify \arg{Val} with the current value of the flag \arg{FlagName}.

    \predicate[det]{get_klog_flag}{3}{+Who:atom, +FlagName:atom, -Val:atom}
Unify \arg{Val} with the current value of the flag \arg{FlagName} of C++ object \arg{Who}.

    \predicate[det]{klog_flags}{1}{+Stream}
Write a description of current flags to \arg{Stream}.

    \predicate[det]{klog_flags}{0}{}
Write a description of current flags to current output stream.
\end{description}


% This LaTeX document was generated using the LaTeX backend of PlDoc,
% The SWI-Prolog documentation system



\section{repair.pl: Referential integrity constraints}

\label{sec:repair}

\begin{tags}
    \tag{See also}
\verb$klog_flag$ \verb$referential_integrity_repair$
\end{tags}

A referential integrity constraint violation occurs if an R-tuple
refers to an undefined identifier (i.e. there is no E-tuple identified
by this foreing key). Violations can be repaired or ignored.\vspace{0.7cm}

\begin{description}
    \predicate[det]{check_and_repair}{1}{+Ex:atom}
Scan the database for possible referential integrity violations and
repair them. The repair strategy is defined by the flag
\verb$referential_integrity_repair$. If the value is \const{add}, then the
defective E-tuple is added. If the value is \const{delete} then the
offending R-tuple is deleted. If the value is \const{ignore} an exception
is raised: simply don't call this predicate if no check should be
performed.
\end{description}



\chapter{Advanced components}
% This LaTeX document was generated using the LaTeX backend of PlDoc,
% The SWI-Prolog documentation system



\section{timing.pl}

\label{sec:timing}
% This LaTeX document was generated using the LaTeX backend of PlDoc,
% The SWI-Prolog documentation system



\section{goals.pl: goals}

\label{sec:goals}

\begin{tags}
    \tag{author}
Paolo Frasconi, taking inspiration from various places
\end{tags}

Module for manipulating sequences of goals like lists.\vspace{0.7cm}

\begin{description}
    \predicate[nondet]{rmember}{2}{?A, ?B}
True when \arg{B} is a comma-separated sequence and \arg{A} occurs in
it. Similar to \predref{member}{2} for lists.

    \predicate[det]{rabsent}{2}{+A, +B}
True when \arg{B} is a comma-separated sequence and \arg{A} does not occur in
it.

    \predicate[det]{rappend}{3}{+A, +B, +C}
True if \arg{A}, \arg{B}, and \arg{C} are comma-separated sequences and \arg{C} is the
concatenation of \arg{A} and \arg{B}. Useful to generate "negative" edges
Warning: does not terminate if \arg{A} is a free variable.

    \predicate[det]{goals_to_list}{2}{+G, -L}
True if \arg{G} is a comma-separated sequence, and \arg{L} the corresponding
list of items.
Warning: does not terminate if \arg{G} is a free variable.

    \predicate[det]{list_to_goals}{2}{+L, -G}
True if \arg{L} is a list of items and \arg{G} the corresponding comma-separated
sequence Warning: does not terminate if \arg{L} is a free variable.
\end{description}


% This LaTeX document was generated using the LaTeX backend of PlDoc,
% The SWI-Prolog documentation system



\section{utils.pl}

\label{sec:utils}

\begin{description}
    \predicate{aformat}{3}{-Atom, +Format, +List}
similar to sformat but builds an atom instead of a list of char codes.

    \predicate[det]{new_progress_bar}{2}{+BarName:atom, +MaxCount:integer}
create a progress bar called \arg{BarName}. The gauged task is completed
when the counter for the bar reaches the integer value \arg{MaxCount}

    \predicate[det]{progress_bar}{1}{+BarName:atom}
increase the counter for the progress bar \arg{BarName} and maybe display
information on screen.

    \predicate[det]{slice}{4}{?L1, +I, +K, ?L2}
\arg{L2} is the list of the elements of \arg{L1} between index \arg{I} and index \arg{K}
(both included).

    \predicate[det]{shuffle}{2}{+Ex:list, -Ex1:list}
Unify \arg{Ex1} with a random permutation of list \arg{Ex}.
\end{description}


% This LaTeX document was generated using the LaTeX backend of PlDoc,
% The SWI-Prolog documentation system



\section{doc/cinterface.pl: C++ interface}

\label{sec:cinterface}

General mechanism: functions listed here are the interface to
Prolog. All model-related predicates take an atom as first argument
that identifies the model (valid atoms should be strings, not
integers). The identifier is then used as a key for a dictionary
stored in the singleton The_ModelPool with associates the model
name to a pointer to an object of Model class (Model is the
abstract class from which all Models are inherited). Typically,
functions defined here should call a homologous method of the class
ModelPool that will dispatch the message to the actual model.

\subsection{Example}

Prolog code in the kLog script:

\begin{code}
new_model(my_model,libsvm_c_svc),
\end{code}

in c_interface, c_new_model() calls the method of the singleton The_ModelPool

\begin{code}
The_ModelPool.new_model("my_model","libsvm_c_svc")
\end{code}

which eventually creates an object of class
Libsvm_Model$<$BinaryClassifierReporter$>$. now back to the prolog
code, the atom my_model can be used as a handle for referring to
the C++ object just created, e.g.

\begin{code}
kfold(target_relation, 10, my_model, my_fg)
\end{code}

that causes the invocation of the method

\begin{code}
The_ModelPool.train("my_model",vector_of_interpretation_ids)
\end{code}

A similar mechanism is used for feature generators via the
singleton The_FeatureGeneratorPool of class FeatureGeneratorPool.

\subsection{Data}

Data is contained in the singleton The_Dataset of type
Dataset. There is a graph for each interpretation and possibly
several sparse vectors associated to the scalar predictions (called
cases in kLog). The former are accessed from prolog via
interpretation identifiers, the latter via "case" identifiers.

The_Dataset maintains internal dictionaries (indices) to
associate identifiers to graph pointers or sparse vector pointers.

Additionally, The_Dataset maintains a map from interpretation ids
to a set$<$string$>$ of case ids. Use The_Dataset.get_cases() to
retrieve this map.\vspace{0.7cm}

\begin{description}
    \predicate[det]{set_c_klog_flag}{3}{+Client:atom, +Flag:atom, +Value:atom}
Sets C-level kLog \arg{Flag} to \arg{Value} for \arg{Client}

    \predicate[det]{get_c_klog_flag}{3}{+Client, +Flag, -Value}
Unifies \arg{Value} with the current value of \arg{Flag} in \arg{Client}.

    \predicate[det]{document_klog_c_flags}{2}{+Client:atom, -Documentation:atom}
Retrieve \arg{Documentation} for all flags belonging to \arg{Client}.

    \predicate[nondet]{klog_c_flag_client}{1}{?Client}
Backtrack over the IDs of the C language flag clients.

    \predicate[nondet]{klog_c_flag_client_alt}{1}{?Client}
Backtrack over the IDs of the C language flag clients.
This implementation uses less memory, it doesn't store a copy of all alternatives in memory

    \predicate[det]{c_shasum}{2}{+Filename, -HashValue}
Unifies \arg{HashValue} with the SHA1 hash value of the file (which must exist).

    \predicate[det]{add_graph}{1}{+Id}
Create a new graph identified by \arg{Id}. \arg{Id} can be any valid Prolog
atom.

    \predicate[det]{add_graph_if_not_exists}{1}{+Id}
Create a new graph identified by \arg{Id} unless it already exists. \arg{Id}
can be any valid Prolog atom

    \predicate[det]{delete_graph}{1}{+Id}
Delete graph identified by \arg{Id}.

    \predicate[det]{write_graph}{1}{+Id}
Write graph identified by \arg{Id} on the standard output.

    \predicate{export_graph}{3}{+Id, +Filename, +Format}
Save graph identified by \arg{Id} into \arg{Filename} in specified
\arg{Format}. File extension is automatically created from format (it is
actually identical). Valid formats are dot, csv, gml, gdl, gspan.

    \predicate[det]{save_as_libsvm_file}{1}{+Filename}
Save a sparse vector dataset into \arg{Filename} in libsvm format. Data
is generated for all interpretations in the attached dataset. Every
line is a case. The line is ended by a comment giving the case
name, which is formed by concatenating the interpretation name and
the identifiers contained in the target fact.

    \predicate[det]{cleanup_data}{0}{}
Clean completely graphs and sparse vectors.

    \predicate[det]{clean_internals}{1}{+GId}
Clean up internal data structures on \arg{GId} to recover memory

    \predicate[det]{vertex_alive}{3}{+GId, +VId, ?State}
Set/get aliveness \arg{State} (yes/no) of vertex \arg{VId} in graph \arg{GId}

    \predicate[det]{set_signature_aliveness}{3}{+GId, +S, +State}
Set aliveness state to \arg{State} to every vertex of signature \arg{S} in graph \arg{GId}

    \predicate[det]{set_slice_aliveness}{3}{+GId, +Slice, +State}
Set aliveness state to Status to every vertex in slice \arg{Slice} in graph \arg{GId}

    \predicate[det]{set_signature_in_slice_aliveness}{4}{+GId, +S, +SliceID, +State}
Set aliveness state to \arg{State} to every vertex of signature \arg{S} in
slice \arg{SliceID} in graph \arg{GId}

    \predicate[det]{set_all_aliveness}{2}{+GId, +State}
Set aliveness state to \arg{State} in every vertex in graph \arg{GId}

    \predicate[det]{add_vertex}{8}{+GId, +SliceId, +Label, +IsKernelPoint, +IsAlive, +Kind, +SymId, -Id}
Create new vertex in graph \arg{GId} with label \arg{Label} and unify \arg{Id} with
the identifier of the new vertex Set KernelPoint to 'yes' to
indicate that this vertex is a kernel point Set \arg{IsAlive} to 'yes'
to indicate that this vertex is alive (i.e. actually exists). Dead
vertices are used for structured output predictions. \arg{Kind} should
be either 'r' for relationship or 'i' for an entity set. \arg{SymId} is
the symbolic id of the vertex and is only used for
visualization/debugging purposes. \arg{SliceId} is the slice to which
this vertex belongs to.

    \predicate[det]{add_edge}{5}{+GId, +V, +U, +Label, -Id}
Create new edge between \arg{U} and \arg{V} with label \arg{Label} and unify \arg{Id} with
the identifier of the new edge

    \predicate[det]{make_sparse_vector}{6}{+ModelType:atom, +FG:atom, +IntId:atom, +SliceId, +CaseId:atom, +ViewPoints:list_of_numbers}
Use feature generator \arg{FG} to make a sparse vector from graph Ex and
focusing on the list of vertices in \arg{ViewPoints}. The sparse vector
is then identified by \arg{CaseId}. \arg{ModelType} is used to construct an
appropriate internal representation of the feature vector.

    \predicate[det]{write_sparse_vector}{1}{+CaseId}
Write the sparse vector identified by \arg{CaseId}.

    \predicate[det]{add_feature_to_sparse_vector}{3}{+CaseId, +FeatureStr, +FeatureVal}
Adds new feature to sparse vector identified by \arg{CaseId}. \arg{FeatureStr}
is hashed to produce the index in the sparse vector. \arg{FeatureVal} is
then assigned to vector[hash(\arg{FeatureStr})]
This method was introduced for ghost signatures

    \predicate[det]{set_target_label}{2}{+CaseID, +Label}
Set the label for SVM \arg{CaseID}

    \predicate[det]{set_label_name}{2}{+Label:atom, +NumericLabel:number}
Record label name

    \predicate[det]{set_rejected}{1}{+CaseID}
Set reject flag for SVM \arg{CaseID}

    \predicate[det]{clear_rejected}{0}{}
Clear all reject flags

    \predicate[det]{remap_indices}{0}{}
Remap feature vector indices in a small range

    \predicate[det]{print_graph_ids}{0}{}
Print all IDs for graphs in the dataset

    \predicate[det]{new_feature_generator}{2}{+FGId, +FGType}
Create a new feature generator identified by atom \arg{FGId}

    \predicate[det]{delete_feature_generator}{1}{+Id}
Delete feature_generator identified by \arg{Id}.

    \predicate[det]{new_model}{2}{+ModelId, +ModelType}
Create a new model identified by atom \arg{ModelId}

    \predicate[det]{delete_model}{1}{+Id}
Delete model identified by \arg{Id}.

    \predicate[det]{model_type}{2}{+ModelId, -ModelType}
Unifies \arg{ModelType} with type of model identified by \arg{ModelId}.

    \predicate[det]{write_model}{1}{+Id}
Write model identified by \arg{Id}.

    \predicate[det]{load_model}{2}{+Id, +Filename}
Load model identified by \arg{Id} from \arg{Filename}

    \predicate[det]{save_model}{2}{+Id, +Filename}
Save model identified by \arg{Id} into \arg{Filename} in kLog internal format

    \predicate[det]{check_model_ability}{2}{+ModelId:atom, +Ability:atom}
Ask \arg{ModelId} whether it can perform the task specified in \arg{Ability}.

    \predicate[det]{train_model}{2}{+ModelId, +DatasetSpec}
Train model identified by ModelIdId. The list \arg{DatasetSpec} should
contain identifiers of interpretations. Case identifiers are then
generated in Pool.h

    \predicate[det]{test_dataset}{3}{+ModelId:atom, +DatasetSpec:atom, +NewData:atom}
Test model identified by ModelIdId on a dataset. The list
\arg{DatasetSpec} should contain identifiers of interpretations. Case
identifiers are then generated in Pool.h. If \arg{NewData} is "true" (or
"yes") then it is assumed that the evaluation is on test data and
results are accumulated in the global reporter of the
model. Otherwise it is assumed that we are testing on training data
and only the local reporter is affected.

    \predicate[det]{dot_product}{3}{+CaseId1, +CaseId2, -Value}
Unify \arg{Value} with the dot product of the feature vectors attached to
case identifiers \arg{CaseId1} and \arg{CaseId2}.

    \predicate[det]{set_model_wd}{2}{+ModelId, +WorkingDir}
Inform model of the working directory. This is mainly needed for
external models in order to keep a clean filesystem.

    \predicate[det]{reset_reporter}{2}{+ModelId, +Reporter}
Reset a reporter of the model. \arg{Reporter} should be either local or
global. 

    \predicate[det]{reset_reporter}{3}{+ModelId, +Reporter, +Description}
Reset a reporter of the model. \arg{Reporter} should be either local or
global. Also sets the description message to \arg{Description}.

    \predicate[det]{report}{2}{+ModelId, +Reporter}
Ask a performance reporter to write a report on the
standard output. \arg{Reporter} should be either local or global.

    \predicate[det]{report}{3}{+ModelId, +Filename, +Reporter}
Ask a performance reporter to write a report to \arg{Filename}. \arg{Reporter}
should be either local or global.

    \predicate[det]{save_predictions}{3}{+ModelId, +Dir, +Reporter}
Save predictions stored in a performance reporter in output.log
and maybe output.yyy in \arg{Dir}. Do AUC analysis for binary
classification reporters. \arg{Reporter} should be either local or global.

    \predicate[det]{get_prediction}{3}{+ModelId, +CaseID, -Margin}
Obtain prediction for \arg{CaseID} in the global reporter of model
identified by \arg{ModelId}. WARNING: fails without displaying any message
if arguments are not valid.
\end{description}



\backmatter
\chapterstyle{demo3}

\clearpage
\twocolindex
%\renewcommand{\chaptermark}[1]{}
\renewcommand{\preindexhook}{%
The first page number is usually, but not always, the primary reference to
the indexed topic.\vskip\onelineskip}
\indexintoc

%%%\raggedright  does nasty things to index entries
\printindex
\end{document}
